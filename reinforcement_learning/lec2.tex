%!TEX root = master.tex

\section{Markov Decision process}
This lecture will explain the concepts of Markov decision process, MDP's, the Bellman equation and optimal policy

\subsection{Recap}
For a state to have \textbf{Markov property} means that the state contains all the information that is useful to predict the future. In other words, we don't need the previous states to predict the future, the useful information is stored in the current state. A \textbf{Policy}: $\pi$(a|s) choosing an action a when we are in state s, can be deterministic or stochastic. The \textbf{prediction} gives the future cumulative reward following a policy. We want to find the policy that maximize the cumulative future reward by \textbf{control}. 

\subsection{Markov decision process}
If we begin with assuming that $\mathcal{S}, \mathcal{A} $ and $\mathcal{R}$ have finite numbers of elements then the \textbf{translation probabilities} are given by:

	\begin{equation}
		p(^{\prime},r |s,a) = \text{Pr}\{S_{t+1} = s^{\prime}, R_{t+1} = r | S_t = s, A_t = a \}
	\end{equation}

The Markov property determines the \textbf{dynamics} of the environment, the probability of transitioning to a new state depends only on the current state and action. The \textbf{state transitions}:

	\begin{equation}
		p(s^{\prime} | s,a) = \sum_{r \in \mathcal{R}} p(s^{\prime},r |s,a) =
	\end{equation}

With the \textbf{expected reward }

	\begin{equation}
		r(s,a) = \mathbb{E} [ R_{r+1} | S_t = s, A_t = a ] =  \sum_{r \in \mathcal{R}}^{} \sum_{s^{\prime \in \mathcal{S}}}^{} r p(s^{\prime},r | s,a)  
	\end{equation}

\subsection{Episodic vs continuing tasks}
\begin{itemize}
	\item \textbf{Episodic tasks}
	\begin{itemize}
		\item Has terminating states and the task end in finite time.
		\item When reaching the terminating state the episode stops.
		\item If you reach the terminating state you will stay there forever, receiving no future rewards.
	\end{itemize}
	\item \textbf{Continuing tasks}
	\begin{itemize}
		\item Often not a clear way to divide the task into independent episodes.
		\item No state were the task is done
		\item Must take into account infinitely many future rewards.
	\end{itemize}
\end{itemize}


\subsection{The return}
In a given state we want to maximize the future reward we can receive: $R_{t+1},R_{t+2, ...}$. To make it possible to have non finite number of rewards we introduce the \textbf{discounted reward}

	\begin{wbox}{The discounted reward}
		\begin{equation}
			G_t = R_{t+1} R_{t+1} + \gamma R_{t+2} + \gamma^{2} R_{t+3} + \ldots = \sum_{k=0}^{\infty} \gamma^{k}R_{t+k+1}, \text{where } 0 < \gamma \le 1
		\end{equation}
	\end{wbox}

since we put $\gamma \le 1$ we put less value on future rewards, $\gamma = 0.5 \Rightarrow \gamma^{10} = 0.001, \gamma = 0.9 \Rightarrow \gamma^{10} = 0.35$.

And if $\gamma < 1$ we make sure that $R_t < \overline{R}$ for all t, and then $G_t$ is bounded:

	\begin{equation}
		\sum_{k=0}^{\infty} \gamma^{k}R_{t+k+1} \le \overline{R}\sum_{k=0}^{\infty} \gamma^{k} = \frac{1} {1-\gamma} \overline{R}
	\end{equation}


If we know that the task ends after a finite number of steps we can get away with using non-discounted returns where $\gamma = 1$

\subsection{The state value function}
The state-value function estimates the expected long term rewards the agent will recieve starting from a specific state and following a given policy. Since $S_t$ and $R_t$ are random variables, the return is therefore also a random variable:

	\begin{equation}
		G_t = R_{t+1} + \gamma R_{t+2} + \ldots
	\nonumber
	\end{equation}

We must therefore consider the expected return:

	\begin{wbox}{The state-value function}
	The state-value function $v_\pi (s)$ of a MDP is the expected return starting from the state s and then following the policy $\pi$:

		\begin{equation}
			v_\pi (s) = \mathbb{E}[G_t | S_t = s]
		\end{equation}
	\end{wbox}

The prediction of cumulative reward is computed with $v_\pi(s)$



\subsection{The action-value function}
Another important value function is the \textbf{action-value function} that estimates the expected long-term reward an agent will receive starting from a specific state, taking a specific action and following a given policy.

	\begin{wbox}{The action-value function}
	The action-value function $q_\pi(s,a)$ is the exptected return starting from s, taking action a, and \textbf{then} following a policy $\pi$
		\begin{equation}
			q_\pi (s,a) = \mathbb{E}[G_t | S_t = s, A_t = a]
		\end{equation}
	\end{wbox}

This function is also often called the Q-function. 

\subsection{Bellman equations}
Looking at the reward we can note that:

	\begin{equation}
		G_t = R_{t+1} + \gamma R_{t+2} + \gamma^{2} R_{t+3} + ... = R_{t+1} + \gamma G_{t+1}
	\end{equation}

Hence, the value function satisfies to following equation:

	\begin{equation}
	\begin{aligned}
		v_\pi(s) = \mathbb{E}_\pi [G_t | S_t = s]\\
		= \mathbb{E}_\pi [R_{t+1} + \gamma G_{t+1} | S_t = s] \\
		= \mathbb{E}_\pi [R_{t+1} + \gamma v_\pi(S_{t+1}) | S_t =s ]
	\end{aligned}
	\end{equation}

The value of s is the expected immediate reward plus the discounted expected vlue of the next state. In the same way is action-value function: 

	\begin{equation}
		q_\pi (s,a) = \mathbb{E} [R_{t+1} + \gamma q_\pi(S_{t+1}, A_{t+1} | S_t = s, A_t = a)]
	\end{equation}
	

\begin{wbox}{}
	\begin{equation}
	\begin{aligned}
		v_\pi(s) = \mathbb{E}_\pi [R_{t+1} + \gamma v_\pi(S_{t+1}) | S_t =s ]\\
		q_\pi (s,a) = \mathbb{E} [R_{t+1} + \gamma q_\pi(S_{t+1}, A_{t+1} | S_t = s, A_t = a)] 
	\end{aligned}
	\end{equation}
\end{wbox}

\subsection{The expectation}
The state-value of a state s, is the expected action-value:

	\begin{equation}
		v_\pi(s)= \sum_{a}^{} \pi(a |s) q_\pi(s,a)
	\end{equation}

So for a deterministic policy a = $\pi(s)$ we get $v_\pi(s,\pi(s))$. And given s and a, the immediate reward r and the next state $s^{\prime}$ has probability $p(s^{\prime}, r |s,a)$ so that:

	\begin{equation}
		q_\pi (s,a) = \sum_{r, s^{\prime}}^{} p(s^{\prime},r | s,a) (r + \gamma v_\pi(s^{\prime})) 		
	\end{equation} 


	\begin{wbox}{The Bellman equation for $v_\pi$ and $q_\pi$}
		\begin{equation}
		\begin{aligned}
			v_\pi = \sum_{a}^{} \pi(a |s)q_\pi(s,a) = \sum_{a}^{} \pi(a,s)\sum_{r, s^{\prime}}^{} p(s^{\prime}, r | s, a)[r + \gamma v_\pi(s^{\prime})] \\
			q_\pi(s,a) = \sum_{r, s^{\prime}}^{} p(s^{\prime},r |s,a) [r + \gamma \sum_{a^{\prime}}^{} \pi(a^{\prime} |s^{\prime})q_\pi(s^{\prime},a^{\prime})]
		\end{aligned} 
		\end{equation}
	\end{wbox}

\subsection{Solving the Bellman equation}
Solving the Bellman equation is done by solving a system of linear equation, this is because the dependencies between the different states. Each state $s \in \mathcal{S}$ gives one equation. There is a unique solution that can be expressed analytically. If there is a large number of states, large $\mathcal{S}$,  then there are more efficient ways to solve it with iterative solution. If $p(s^{\prime}, r |s,a)$ is not know, we need to \textbf{learn} $v_\pi(s)$ from \textbf{experience}. If the number of states $\mathcal{S}$ is infinite, we can't compute the value for each state individually, and instead we need to find some function $\hat{v}(s,\textbf{w}) \approx v_\pi(s)$. 


\subsection{Optimal value function}
This represent the maximum expected cumulative reward that can be achieved from a given state, following the best possible policy. It quantifies the highest long-term reward that the agent can expect to obtain when makin \textbf{optimal decision} from each state. The optimal state-value function is given by:

	\begin{equation}
		v_*(s) = \max v_\pi(s), \text{for all } s \in \mathcal{S}
	\end{equation}

The optimal action-value function is given by:

	\begin{equation}
		q_*(s,a) = \max q_\pi(s,a), \text{for all } s \in \mathcal{S} \text{ and } a \in \mathcal{A}
	\end{equation}

The optimal $v_*(s)$ should be the maximum of $q_*(s,a)$, so we have:

	\begin{equation}
		v_*(s) = \underset{a}{\text{max }} q_*(s,a)
	\end{equation}

The equation for $q_*(s,a)$:


	\begin{equation}
	\begin{aligned}
		q_*(s,a) = \sum_{r, s^{\prime}}^{} p(s^{\prime},r |s,a)(r + \gamma v_*(s^{\prime})) = \underset{a}{\text{max }} \mathbb{E}[R_{t+1} + \gamma(S_{t+1}) | S_t = s, A_t = a] \\
		 = \mathbb{E}[R_{t+1} + \gamma \underset{a^{\prime}}{\text{max }} q_* (S_{t+1}, a^{\prime}) |S_t =s, A_t = a]
	\end{aligned}
	\end{equation}


\begin{wbox}{Solving the Bellman optimality function}
	\begin{equation}
		v_*(s) = \underset{a}{\text{max }} \sum_{r,s^{\prime}}^{} p(s^{\prime},s |s,a)[r+\gamma v_*(s^{\prime})]
	\end{equation}
\end{wbox}

This is a system of non-linear equation (non linear because of the max function). There is one equation for each state, s. In general there is no closed form solution. But there are iterative methods to solve it. 

\subsection{Optimal policy}
\textbf{Partial ordering over policies: } This is the relationship between different policies, where some of them can be ranked as better or worse than others. This is based on their performance or expected returns. For some policy pairs there is no such definitive ranking. This means that the ordering captures the hierarchy of policies in terms of effectiveness  but not necessarily a linear ranking of them. 
	\begin{equation}
		\pi \ge \pi^{\prime} \text{ if } v_\pi(s) \ge t_{\pi^{\prime}}(s), \text{ for all }s
	\end{equation}

\begin{wbox}{Theorem}
\begin{itemize}
	\item There exists at least one optimal policy $\pi_*$ such that $\pi_* \ge \pi$ for all policies $\pi$.
	\item All optimal policies achieve the optimal state-value function $v_{\pi *}(s) = v_*(s)$ 
	\item All optimal policies achieve the action-value function $q_{\pi *} (s,a) = q_* (s,a)$
\end{itemize}
\end{wbox}


So how do make a decision in state s?

\begin{enumerate}
	\item Choose an action, a that maximizes the optimal action-value $q_*(s,a)$
	\item Then use an optimal policy form $s^{\prime}$
\end{enumerate}

The \textbf{control} part is to find this optimal policy. Were the policy is described with:

	\begin{equation}
		\pi_* (s) = \arg \underset{a}{\text{max }} \sum_{r, s^{\prime}}^{} p(s^{\prime},r |s,a) [r + \gamma v_*(s^{\prime})] 
	\end{equation}

which is optimal. But \textbf{note} if we already know $q_*(s,a)$ then we don't need the dynamics to find an optimal policy. 





