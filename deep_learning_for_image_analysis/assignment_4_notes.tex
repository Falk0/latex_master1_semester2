
\documentclass[a4paper]{article}
\usepackage[a4paper,left=2.5cm,right=2cm,top=2.5cm,bottom=2.5cm]{geometry}
\usepackage[utf8]{inputenc}
\usepackage[T1]{fontenc}
\usepackage{textcomp}
\usepackage[english]{babel}
\usepackage{amsmath, amssymb}
\usepackage{natbib}
\usepackage{float} 
\usepackage[caption = false]{subfig}
\usepackage{listings}
\lstset{
    breaklines=true,
    basicstyle=\tt\normalsize,
    keywordstyle=\color{blue},
    identifierstyle=\color{magenta},
    frame = single
} 
% figure support
\usepackage{import}
\usepackage{xifthen}
\pdfminorversion=7
\usepackage{pdfpages}
\usepackage{transparent}
\pdfsuppresswarningpagegroup=1

\title{Assignment 4 - DL4IA \\ Explainable Artificial Intelligence}
\author{Linus Falk}
\begin{document}
\maketitle

\section{Introduction}


\subsection{Method and architecture}

\subsection{Applications and advantages}

\subsection{Grad-CAM and its differences from CAM}

\subsection{Challenges and limitations}


\vspace{1cm}


\section{Network and testing}


\subsection{Select Network and source code for testing}


\subsection{Find and Test images within classes of chosen network}
Save attention maps for predicted class and some similar and dissimilar classes



\subsection{Apply one variation to each image, record changes}
Save attention maps for predicted class and some similar and dissimilar classes

\subsection{Team photo}

\section{Summarize}

\subsection*{Discussion}
Discuss the effects of the variations on the attention maps.







\bibliographystyle{unsrt}
\bibliography{references}
\end{document}

Method and architecture:
Understand the main concept of Class Activation Mapping and its underlying method.



