\section{Light microscopy}
short intro of light microscopy ....

\subsection*{History}


\subsection*{Operational steps}


Compound microscope
	Parts

	Optics - 
		Total magnification
		Possible resolution
		Numerical aperture 
		Resolution 
		Oil immersion lenses
		Objectives 
		Airy disc

	Staining
		H\&E staining 
		Immunostaining
			Primary immunostaining
			Secondary immunostaining

	Densitometric - Light transmission, signal type, electromagnetic waves, 
					physical property Transmission - exponentially absorbed - intensity proportional to absorbing matter. Sample contrast comes from attenuation of light in the sample
					Active illumination all at once
	
	Spatial/Geometrical - Projection of transmission through the object. Resolution limited by wavelength and ... oil immersion 

	Spectral - Wavelength interval visible light. 

	Temporal -  still image, no motion to consider




Dark field microscopy
	-Parts

	-Optics
	Oil immersion for best scatter light gathering, Illumination of the sample partly blocked 
	
	-Staining
	allows to increase contrast of images of live unstained and uncolored objects sample makes some of the light scatter even if it's transparent, this light is then refocused on the detector 

	- Densiometric 
	sample contrast comes from light scattered by the sample, Scatterd light produces the 				image, transmitted light not collected. Resulting in an image of the sample superimposed on a dark background. Low light levels in final image. Strongly illuminated sample, can cause damage. Almost totally Halo artifact free technique 

	\textcolor{red}{"Care must be taken when preparing specimens for darkfield microscopy because the features that lie above and below the plane of focus can scatter light and contribute to image degradation."} 

	Diffraction patterns artifact. 


	- Spatial/Geometrical 
	Greater care of interpretation needed since some features may not be visible in darkfield that are visible in brightfield. Darkfield images lacks low spatial frequencies and is a high passed version of the underlying structure. May appear to be a negative of brightfield image but different effects/features are visible in each. 

	- Spectral
	Can be coupled with hyperspectral imaging with the right kind of light source, most commonly is the visible light spectrum

	- Temporal




Phase contrast microscopy

	Parts 
	No staining. \textcolor{red}{"No staining The phase-contrast microscope made it possible for biologists to study living cells and how they proliferate through cell division."}

	-Optics
	\textcolor{red}{"In a phase-contrast microscope, image contrast is increased in two ways: by generating constructive interference between scattered and background light rays in regions of the field of view that contain the specimen, and by reducing the amount of background light that reaches the image plane. First, the background light is phase shifted by - 90 degrees by passing it through a phase-shift ring, which eliminates the phase difference between the background and the scattered light rays. When the light is then focused on the image plane (where a camera or eyepiece is placed), this phase shift causes background and scattered light rays originating from regions of the field of view that contain the sample (i.e., the foreground) to constructively interfere, resulting in an increase in the brightness of these areas compared to regions that do not contain the sample. Finally, the background is dimmed 70 -90 percent by a gray filter ring: this method maximizes the amount of scattered light generated by the illumination light, while minimizing the amount of illumination light that reaches the image plane."} 

	- Densiometric
	Transmission, Sample contrast comes from interference of different path lengths of light through the sample 

	\textcolor{red}{"Phase contrast doesn't work well with thick specimens as these can appear distorted."} 

	-Spatial/Geometrical 
	\textcolor{red}{"Phase contrast images often have halos surrounding the outline of details that have a high phase shift. These halos are optical artefacts and can make it hard to see the boundaries of details."} 
	\textcolor{red}{"The resolution of phase images can be reduced due to the phase annuli limiting the numerical aperture of the system."}  


	-Spectral

	-Temporal



Differential interference contrast microscopy
	Parts

	- Densiometric
	\textcolor{red}{"Transmission, Contrast created due to difference in path length for two adjacent points in sample. Image small differences in refractive index between neighboring points, Image small differences in refractive index between neighboring points"} 
	\textcolor{red}{"The technique produces a monochromatic shadow-cast image that effectively displays the gradient of optical paths for both high and low spatial frequencies present in the specimen. Those regions of the specimen where the optical paths increase along a reference direction appear brighter (or darker), while regions where the path differences decrease appear in reverse contrast. As the gradient of optical path difference grows steeper, image contrast is dramatically increased."} 
	-Spatial/Geometrical 
	\textcolor{red}{advantage: "There are no confusing halos as may be encountered in phase images."} 
	-Spectral

	-Temporal



Confocal microscopy
	Parts

	\textbf{-Densiometric}
	Contrast created 
	\textcolor{red}{"a confocal microscope only focuses a smaller beam of light at one narrow depth level at a time. The CLSM achieves a controlled and highly limited depth of field. "} 

	\textcolor{red}{"confocal microscope uses point illumination (see Point Spread Function) and a pinhole in an optically conjugate plane in front of the detector to eliminate out-of-focus signal – the name "confocal" stems from this configuration. As only light produced by fluorescence very close to the focal plane can be detected, the image's optical resolution, particularly in the sample depth direction, is much better than that of wide-field microscopes. However, a"} 

	\textcolor{red}{"Confocal microscopy provides the capacity for direct, noninvasive, serial optical sectioning of intact, thick, living specimens with a minimum of sample preparation as well as a marginal improvement in lateral resolution compared to wide-field microscopy.[4] Biological samples are often treated with fluorescent dyes to make selected objects visible. However, the actual dye concentration can be low to minimize the disturbance of biological systems:"} 


	\textbf{-Spatial/Geometrical} 
	\textcolor{red}{"The achievable thickness of the focal plane is defined mostly by the wavelength of the used light divided by the numerical aperture of the objective lens, but also by the optical properties of the specimen. The thin optical sectioning possible makes these types of microscopes particularly good at 3D imaging and surface profiling of samples."} 


	\textbf{-Spectral}


	\textbf{-Temporal}
	\textcolor{red}{"s much of the light from sample fluorescence is blocked at the pinhole, this increased resolution is at the cost of decreased signal intensity – so long exposures are often required."} 



Reflected light microscopy

	-Densiometric
	Contrast created 


	-Spatial/Geometrical 

	-Spectral

	-Temporal



Fluorescence microscopy

	-Densiometric
	Contrast created 


	-Spatial/Geometrical 

	-Spectral

	-Temporal


\subsection*{Densitometric}



\subsection*{Geometrical}

\subsection*{Spectral}

\subsection*{Temporal}

