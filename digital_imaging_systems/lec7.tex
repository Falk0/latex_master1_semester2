\section{PET, CT and MRI}
All of these techniques are so called tomography techniques (imaging slices of the sample). They are used in both clinical routine and in research. In common that they are: expensive, risks involved in using them (Radioactivity, Ionizing radiation and strong magnetic fields). The geometrical accuracy is good for all of them with some exceptions for MRI. Can perform both static and dynamic imaging (time series). They share the challenges of tunnel size (PET/MRI and MRI). Much development is ongoing, PET tracers, hardware and software. 

	\subsection*{PET - Positron Emission Tomography}
	Uses a \textbf{tracer} or radiopharmaceutical that is emitting positrons. This radionuclide isotope is attached on a molecule of interest, glucose for an example in Fluorodeoxyglucose $^{18}F$. The doses are small (microdosing) and there is no expected pharmacological effect of the patient. The PET systems create an image of the concentration of tracer. There are many different types of tracers, and more under development. Depending on what is needed to be imaged or what is being looked for the tracer is chosen. The images are typically of quite low resoluition and don't give much anatomical information, it is therefore most often combined with MRI or CT. 

	\subsection*{Workflow}
	The unstable isotope is created in a cyclotron often close to the PET facility, since the halftime is often short. It is then combined with the molecule of interest and we have the tracer that can be injected into or in some cases inhaled by the patient. The tracer emits positrons (beta particles) that can only travel short distances (mm) before it annihilates with an electron and create two 2 gamma rays shooting of approx 180 degrees apart. The detector rings picks up these gamma rays (Coincidence detection) and counts them, giving a line of response (LoR). This results in \textbf{Sinogram} 