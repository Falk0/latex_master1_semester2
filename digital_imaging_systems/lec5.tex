\section{Transmission Electron Microscopy - TEM}
The big data in TEM. A sample is only a couple of mm big and not the whole sample is examined, just tiny parts of it. It is therefore important to remember that what we are observing in TEM is only a small small part of the sample. If we were to examine the whole sample (3mm in diameter) it would generate approximately 29Tb of data. 

	\subsection*{What is it used for?}
	When we want to observe something at nanometer level the only way is to use electron microscopy. Other methods may offer some quantitative information but for looking directly we need electron microscopy. Some examples of applications are:

	\begin{itemize}
		\item Biology \begin{itemize}
			\item Drug development and quality control
					\item Diagnosis
					\item Treatment planning
					\item Understanding diseases
					\item localization
					\item Protein structures, create 3D maps
		\end{itemize}
		\item Material science \begin{itemize}
					\item Chemical compositions
					\item Structures
					\item Characterization
					\item Quality control and material development
		\end{itemize}
	\end{itemize}

	\subsection*{TEM systems are bulky}
	TEM systems are bulky and expensive systems that need expert skilled personnel to prepare the samples and operate the equipment. The facilities are adapted for the system and need many restrictions. Separate cooling and 380-400V 3 phase need for the system, makes it hard to place a system without careful planning. Sensitivity to vibration makes need for special flooring. The maintenance  of the system of expensive requiring planned down time. Unplanned downtime increases the cost of the system. With the system comes also the need for careful waste handling since many samples can be toxic or covered with radioactive salts that are typically used to protect the samples during exposure. 


	\subsection*{The anatomy of a TEM system}
	Modern TEM system is looking more like a fridge/freezer combo than the previous generations that are illustrated below. Much similarity to an optical microscope, the lenses are replaced with magnetic lenses but the principle stays the same. Field emission is the most common electron source. 


	\subsection*{Types of TEM}
	Since the principle stays the same we have Brightfield and Darkfield imaging in TEM also. In Darkfield TEM we get enhancement of inorganic NPs that are easily hidden in Brightfield, Brightfield more used in biology. Mass contrast is another type of imaging from TEM that we dont see in optical microscopy. With mass contrast we can see contrast between regions consisting of different z-number. It is the objective aperture that creates the mass/z-contrast. Low mass gets more electrons through, more mass scatter more electrons which gives contrast. 

	\subsection*{Sample preparation}
	Sample preparation is very important since we are blasting electron through the sample to create an image. To to thick sample will now allow any electrons to pass through. Preparing a sample of a hard material often involves making a wedge shaped piece of the material where the thinnest part which is used need to be 5 to a few 100 nanometers thick.  
	  



	      
