%!TEX root = master.tex
\section{Transmission Electron Microscopy - TEM}
The big data in TEM. A sample is only a couple of mm big and not the whole sample is examined, just tiny parts of it. It is therefore important to remember that what we are observing in TEM is only a small small part of the sample. If we were to examine the whole sample (3mm in diameter) it would generate approximately 29Tb of data. 

	\subsection{What is it used for?}
	When we want to observe something at nanometer level the only way is to use electron microscopy. Other methods may offer some quantitative information but for looking directly we need electron microscopy. Some examples of applications are:

	\begin{itemize}
		\item Biology \begin{itemize}
			\item Drug development and quality control
					\item Diagnosis
					\item Treatment planning
					\item Understanding diseases
					\item localization
					\item Protein structures, create 3D maps
		\end{itemize}
		\item Material science \begin{itemize}
					\item Chemical compositions
					\item Structures
					\item Characterization
					\item Quality control and material development
		\end{itemize}
	\end{itemize}

	\subsection{TEM systems are bulky}
	TEM systems are bulky and expensive systems that need expert skilled personnel to prepare the samples and operate the equipment. The facilities are adapted for the system and need many restrictions. Separate cooling and 380-400V 3 phase need for the system, makes it hard to place a system without careful planning. Sensitivity to vibration makes need for special flooring. The maintenance  of the system of expensive requiring planned down time. Unplanned downtime increases the cost of the system. With the system comes also the need for careful waste handling since many samples can be toxic or covered with radioactive salts that are typically used to protect the samples during exposure. 


	\subsection{The anatomy of a TEM system}
	Modern TEM system is looking more like a fridge/freezer combo than the previous generations that are illustrated below. Much similarity to an optical microscope, the lenses are replaced with magnetic lenses but the principle stays the same. Field emission is the most common electron source. 


	\subsection{Types of TEM}
	Since the principle stays the same we have Brightfield and Darkfield imaging in TEM also. In Darkfield TEM we get enhancement of inorganic NPs that are easily hidden in Brightfield, Brightfield more used in biology. Mass contrast is another type of imaging from TEM that we dont see in optical microscopy. With mass contrast we can see contrast between regions consisting of different z-number. It is the objective aperture that creates the mass/z-contrast. Low mass gets more electrons through, more mass scatter more electrons which gives contrast. 

	\subsection{Sample preparation}
	Sample preparation is very important since we are blasting electron through the sample to create an image. To to thick sample will now allow any electrons to pass through. Preparing a sample of a hard material often involves making a wedge shaped piece of the material where the thinnest part which is used need to be 5 to a few 100 nanometers thick. For \textbf{thin film} samples are the thin film of interest placed on top of a Ni/Ti film which sits on top of a Si substrate. Then is the process (similar with cross sections samples):

		\begin{enumerate}
		  	\item Grind away some of the substrate 
		  	\item Make a dimple in the substrate
		  	\item Bombard the dimple with ions to make it thinner and brake the sample into two pieces
		  	\item Image the sample
		\end{enumerate}  
	  
	Stain such as heavy metals salts can be used to cover biological samples that otherwise would collapse of the electron beam. Gives the negative image of the object since we stain the background, improves contrast against the background to the biological sample that is otherwise poor. Unstained specimen is drying and suffers of deterioration which also affect the image. Embedding the specimen with the stain preserves the integrity of the sample. 

	\subsection{nsTEM workflow}
	When working with liquid biological samples the workflow could look like this: Electron microscope grid is placed in a \textbf{Glow discharger} which makes the liquid sample able to cover the grid more evenly (reduces surface tension \textcolor{red}{?}). The sample is placed on the grid with the excess blotted away, a wash of ultrapure water is applied with excess blotted away after which the stain is applied (excess blotted away). This process requires skill and patience, doing it once might not be that difficult but repeating it equally good each time is an art. Some problems that could arise are: too much or to little stain, dehydrated grid areas, stain deposits or stain crystals.

	\subsection{Ultrathin biological section preparation}
	When working with biological samples like cells it is important to work fast since it start to deteriorate as soon it comes contact with air. It is therefore fixed with \textbf{chemical fixation} after which a heavy metal \textbf{stain} is used to increase contrast, it is then \textbf{dehydrated} and \textbf{resin embedded}. A cell is typically to large to be observed in a TEM whole and must be sliced down using \textbf{ultramicrotomy}. When sliced it can be stained using \textbf{antibody} stain in combination with heating if needed. A secondary antibody stain connected with gold is then applied before putting it into the TEM system. 

	\subsection{CryoTEM workflow}
	When using cryoTEM it is possible to prepare and observe the sample without using toxic heavy metal stains. The process of preparing the sample starts as the previous with placing the EM holder in a \textbf{glow discharger} to make the sample spread better on the grid. Instead of applying the stains it is plunged freezed (to avoid crystals forming). After it is stored in different ways until its observed in a cryoTEM. 

	\subsection{Pros and cons nsTEM and cryoTEM}
	\textbf{nsTEM}
	\begin{itemize}
		\item \textcolor{OliveGreen}{Straight forward method}
		\item \textcolor{OliveGreen}{Allow to observe particles in diluted samples} 
		\item \textcolor{OliveGreen}{Debris and background easy to identified}
		\item \textcolor{OliveGreen}{Grid/sample can be observed at multiple occasions}
		\item \textcolor{red}{Discrimination between filled and empty particles poorly reliable, "don't see inside"}
		\item \textcolor{red}{Preparation of specimen may alter particle morphology and integrity}
		\item \textcolor{red}{Artifacts from preparations common to observe}  
	\end{itemize}

	\textbf{cryoTEM}
	\begin{itemize}
		\item \textcolor{OliveGreen}{Specimen is observed close to its native state}
		\item \textcolor{OliveGreen}{Possible to observe internal parts of specimen}
		\item \textcolor{OliveGreen}{Discrimination between empty and filled particles good}
		\item \textcolor{OliveGreen}{High resolution of structural information}
		\item \textcolor{red}{Tedious preparation}
		\item \textcolor{red}{The grid can only be imaged once}
		\item \textcolor{red}{Most work with low electron dosage to avoid damaging the specimen}       
	\end{itemize}
	  
	
























	    


	      
