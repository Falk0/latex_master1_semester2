\section{SEM - Scanning Electron Microscopy}

\subsection*{Description of SEM}
The electron microscopy is an instrument that allow us to create images/visualize organic and inorganic structures with impressive magnification.It is an invaluable instrument in the engineering and development of new materials where nano-meter sized imaging is very important. The SEM can also be used in chemical analysis and accessing the crystalline structures of materials.  


In comparison to light microscopy is the depth of field much larger in scanning electron microscopy thanks to the narrow electron beam. The resulting images get a 3D appearance due to this large field of depth that is very useful when examining the surface structures. 





\subsection*{Operational steps}
The idea behind the technique is to generate a beam of energetic electrons with emission from an electron source. This is typically done by heating up a metal in vacuum and accelerate the electrons with a electric field. The electrons are stopped from leaving the atoms of the metal be an energy barrier between the emission tip and the surrounding vacuum. By adding an electric field in the vacuum the energy barrier is reduced to a "slope", but the electrons still need to perform the "work" W to pass this barrier. Luckily there is an quantum effect called tunneling allowing electrons to tunnel through this barrier out into to vacuum.   

\textcolor{red}{Picture of tunneling} 

The energy of the electrons in this beam is measured in 1eV which is equivalent to what a electron in an electric field generated by 1 V would have. This energy is denoted $E_0$ and is often around 0.1 to 30keV. The electron beam is after acceleration modified/reshaped by lenses and apertures to reduce the diameter of the beam and to scan this beam into a raster of x-y coordinates which it sequentially is placed in. This coordinates are discrete but closely spaced. The lenses are no ordinary glass lenses since the electrons would pass through them or lose to much energy (wouldn't focus them either). Instead are they modified with electrostatic lenses and electromagnetic coils into desired shape and properties. 


At each of the raster coordinates in the scan pattern are two types of outgoing electrons from the specimen created: \textbf{back-scattered electrons (BSE's)} and \textbf{secondary electrons (SE's)}. The BSE's are electrons that emerge from the specimen with a lot of the initial energy after interacting with the atoms electric fields of the atoms in the specimen, scattering and deflection. The SE's are electrons that emerge from the specimen surface after the electron beam have ejected them from the atoms in the sample. These electron escapes with very low energy compered to the typically high energy electron beam. They are in the range of 0-50eV with the majority below 5eV. 


Electron-sample interaction, the interaction volume. 

The secondary electrons are often measured with a Everhart-Thornley detector that is sensitive for both SE's and BSE's while the BSE's are measured with a dedicated BSE detector that is not sensitive for the SE's. The signal for each of the detectors are measured at each of the coordinates in the x-y raster. The location and the intensity is recorded and correspond to a gray value in the x-y coordinates. Other sensors can also be used to capture for example X-ray signals that also can be emitted from the specimen due do the electron beam.


\subsection*{Densitometric}
In order to create contrast between the different features of the sample there is need for a signal need to distinguish them. The BSE's and SE's give us different information about the specimen.

The secondary electron energy 

\subsection*{Geometrical}

\subsection*{Spectral}

\subsection*{Temporal}

\subsection*{History}




