

\documentclass[a4paper]{article}

\usepackage[utf8]{inputenc}
\usepackage[T1]{fontenc}
\usepackage{textcomp}
\usepackage[english]{babel}
\usepackage{amsmath, amssymb}
\usepackage{natbib}
\usepackage{float}
\usepackage{url}
\usepackage[caption = false]{subfig}
\usepackage{listings}
\lstset{
    breaklines=true,
    basicstyle=\tt\normalsize,
    keywordstyle=\color{blue},
    identifierstyle=\color{magenta},
    frame = single
} 
% figure support
\usepackage{import}
\usepackage{xifthen}
\pdfminorversion=7
\usepackage{pdfpages}
\usepackage{transparent}
\pdfsuppresswarningpagegroup=1

\title{Test title}
\author{Linus Falk}
\begin{document}
\maketitle

% ------------------------------------------------------------------
\section{Introduction}
\url{https://www.imechyperspectral.com/en/applications/hyperspectral-remote-sensing}
\url{https://www.youtube.com/watch?v=RZu1LHumbiQ NC state university}

    \subsection*{What is it and what is it used for- images of system and example 
    images the system creates
    }

    Three articles to rule them all: 
    \begin{itemize}
        \item \url{https://www.enmap.org/data/doc/Science_Plan_EnMAP_2022_final.pdf}
        \item \url{https://ieeexplore.ieee.org/stamp/stamp.jsp?tp=&arnumber=9463743}
        \item \url{https://eo-college.org/resource-spectrum/hyperspectral/}
    \end{itemize}


% ------------------------------------------------------------------
\section{Brief history}
%satelite history
\url{https://www.nasa.gov/topics/earth/features/eo1-10th.html}



    \subsection*{Who/when was it developed and for what purpose}
    \url{https://www.photonics.com/Articles/Measuring_the_Earth_from_Above_30_years_and/a47298}
    \url{https://nij.ojp.gov/topics/articles/hyperspectral-imaging-and-search-humans-dead-or-alive}
    \url{https://www.sciencedirect.com/science/article/abs/pii/S003442570900073X}

    \subsection*{How has it evolved}

% ------------------------------------------------------------------

\section{Use today \& limiting factors}

\url{https://gisgeography.com/hyperspectral-imaging/}
\url{https://hyperspectral.azavea.com/}



    \subsection*{Is it expensive, bulky, easy to use, can anyone use it}


    \subsection*{Safety aspects, waste (toxic chemical, environmental effects)}

% ------------------------------------------------------------------
\section{How is the image formed}

    \subsection*{How is the instrument constructed}
    %sensor types
    \url{https://www.mdpi.com/1424-8220/19/14/3071#B26-sensors-19-03071}

    \url{https://archive.ll.mit.edu/publications/journal/pdf/vol15_no2/15_2-08.pdf}

    \url{https://www.eoportal.org/satellite-missions/eo-1#hyperion}

    \subsection*{What is the energy source}


    \subsection*{What signal is detected and how}



% ------------------------------------------------------------------
\section{What property of the sample is imaged?}
    \url{https://archive.ll.mit.edu/publications/journal/pdf/vol15_no2/15_2-08.pdf}

    \subsection*{Chemical property or density or something else}

    \subsection*{Is the sample prepared somehow}

    \subsection*{If so, how does that influence the image (limit resolution, change 
    what property that is imaged, etc.)}


% ------------------------------------------------------------------
\section{Resolution and sample size}

    \subsection*{What is the typical resolution in the different dimension (x,y,z,t, etc.)}
        
        Spatial resoltuion reduced compared to multispectral, hard or impossible to have many bands and high resolution
        
        \url{https://ieeexplore.ieee.org/abstract/document/978246} 
        resolution 30m ground, 

    \subsection*{What limits the resolution}

    \subsection*{What sample size is typically imaged }


% ------------------------------------------------------------------
\section{Cost and limiting factors}

    \subsection*{How much does a system cost?}
        \url{https://spaceflightnow.com/2022/03/31/german-imaging-satellite-gets-top-billing-on-next-spacex-rideshare-launch/}
    \subsection*{What level of expertise is required by the user? Can anyone use it?}
        \url{https://www.dlr.de/eoc/en/desktopdefault.aspx/tabid-5514/20470_read-47899/}
    \subsection*{What factors are limiting its use?}


% ------------------------------------------------------------------
\section{Variants and future use}

\url{https://www.esa.int/Enabling_Support/Space_Engineering_Technology/Hyperspectral_imaging_by_CubeSat_on_the_way}

\url{https://www.mdpi.com/1424-8220/19/14/3071#B26-sensors-19-03071}

\url{https://www.imechyperspectral.com/en/applications/hyperspectral-remote-sensing}

    \subsection*{Any special variants worth mentioning}

    \subsection*{Where is the imaging technique heading? }

    \subsection*{Where do you see it used in the future?}




\bibliographystyle{unsrt}
\bibliography{references}
\end{document}

