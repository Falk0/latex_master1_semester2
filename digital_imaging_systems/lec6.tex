\section{Ultrasound}
Ultrasound (and all other types of sounds) are acoustic waves, i.e. a mechanical perturbation traveling in a medium. The acoustic frequency range:

\begin{itemize}
	\item Infrasound: < 20 Hz
	\item Audible sound: 20 Hz - 20 kHz
	\item Ultrasound: > 20 kHz 
	\item (Medical ultrasound: 1 - 20 MHz) 
\end{itemize}

Ultrasound is therefore mechanical oscillations with a frequency above 20 kHz that propagates through a medium that is elastic. Used for finding fungus in trees, used by bats to find their prey and in quality control of weld joint for finding cracks to give a few examples. The discovery of the piezoelectric property made this technique possible. 

	\subsection*{Ultrasound in medical imaging}
	Ultrasound is one of the most used imaging techniques in medicine and have an annual market growth of 3-4\%.Pros for this technique are that is non ionizing, need minimal safety requirements, its real time, can be made portable and is quite low cost. The downside with real time is the need of training personnel to understand/interpret the image while making the examination. Signal to noise ratio is quite low compared to other techniques and it need a acoustic window to operate (can't "see through" bone). 

	\subsection*{Probe - probing mechanism}
	Ultrasound is a longitudinal wave, variation of pressure that propagates, this mean that it needs a medium to travel through. The speed of sound is a function of the medium:

		\begin{equation}
			c = \sqrt{\frac{K} {\rho} }		
		\end{equation}

	Where K is the stiffness, stiffer the faster and $\rho$ is density: denser the slower. Like electrical circuits with resistance and current, the propagation of sound can be calculated with the Acoustic impedance:

		\begin{equation}
			Z = \rho c
		\end{equation}

		\begin{table}[ht!]
		\centering
		\begin{tabular}{llll}\hline
		 Material&  Density (kg$\text{m}^{-3}$&  Speed of sound (m$\text{s}^{-3}$)& Acoustic impedance (kg$\text{m}^{-2}$$\text{s}^{-1} \times 10^6$ ) \\
		 Air&  1.3&  330& 0.000429 \\
		 &  & & \\ \hline
		\end{tabular}
		\caption{example}
		\label{tab:tab1}
		\end{table}