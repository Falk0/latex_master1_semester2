\section{Ultrasound}
Ultrasound (and all other types of sounds) are acoustic waves, i.e. a mechanical perturbation traveling in a medium. The acoustic frequency range:

\begin{itemize}
	\item Infrasound: < 20 Hz
	\item Audible sound: 20 Hz - 20 kHz
	\item Ultrasound: > 20 kHz 
	\item (Medical ultrasound: 1 - 20 MHz) 
\end{itemize}

Ultrasound is therefore mechanical oscillations with a frequency above 20 kHz that propagates through a medium that is elastic. Used for finding fungus in trees, used by bats to find their prey and in quality control of weld joint for finding cracks to give a few examples. The discovery of the piezoelectric property made this technique possible. 

	\subsection*{Ultrasound in medical imaging}
	Ultrasound is one of the most used imaging techniques in medicine and have an annual market growth of 3-4\%.Pros for this technique are that is non ionizing, need minimal safety requirements, its real time, can be made portable and is quite low cost. The downside with real time is the need of training personnel to understand/interpret the image while making the examination. Signal to noise ratio is quite low compared to other techniques and it need a acoustic window to operate (can't "see through" bone). 

	\subsection*{Probe - probing mechanism}
	Ultrasound is a longitudinal wave, variation of pressure that propagates, this mean that it needs a medium to travel through. The speed of sound is a function of the medium:

		\begin{equation}
			c = \sqrt{\frac{K} {\rho} }		
		\end{equation}

	Where K is the stiffness, stiffer the faster and $\rho$ is density: denser the slower. Like electrical circuits with resistance and current, the propagation of sound can be calculated with the Acoustic impedance:

		\begin{equation}
			Z = \rho c
		\end{equation}

\newpage

		\begin{table}[ht!]
		\centering
		\begin{tabular}{lllm{5cm}}\hline
		\textbf{Material} & bf\textbf{Density (kg$\text{m}^{-3}$)}& \textbf{Speed of sound}& \textbf{Acoustic impedance \newline (kg$\text{m}^{-2}$$\text{s}^{-1} \times 10^6$ )} \\

		Air & 1.3 &  330& 0.000429\\

		Water & 1000 & 1450& 1.50\\

		Bone & 1500 & 4000& 6.0\\
		
		Blood & 1060 & 1570& 1.59\\

		Muscle (average) & 1075 & 1590& 1.70 \\ \hline
		
		\end{tabular}
		\caption{Example of a table with line breaks}
		\label{tab:example}
		\end{table}

	The \textbf{wavelength} $\begin{bmatrix} m \end{bmatrix}$ is determined by: \textbf{Frequency} $\begin{bmatrix} Hz \end{bmatrix}$ (depends on the source) and \textbf{Sound-speed} $\begin{bmatrix} m/s \end{bmatrix}$ (depends on medium). 

	\subsection*{Matter interaction}
	After the sound wave left the source (probe) it interact with the material with the tissue, being \textbf{back-scattered}, \textbf{reflected}  and \textbf{refracted}. The sound wave is attenuated while moving through the tissue. 

	When the wave travels into a new medium the impedance of the medium and the incidence angle determines what happens to the wave (Huygens principle). If the wavefront is perpendicular to the medium the wavelength will change. If there is an oblique incidence angle it \textbf{refracts} giving a \textbf{reflection} against the new medium and \textbf{transmitted} wave through the new medium. The intensity of each part is dependent on: the \textbf{acoustic impedance difference} between the two materials and the \textbf{angle of incidence}.

	\textbf{Scattering} is the directionless reemission of incident energy caused by local inhomogeneities. Inside tissue, this scattering is caused by "tiny" particles, cells, large proteins, calcifications and so on. The interaction between this scattered waves is what causes the noisy speckle texture pattern in ultrasound images. 

	\textbf{Attenuation} is the loss of energy of the wave in the material. This energy is lost over distance due to \textbf{absorption}, \textbf{scattering}, mode conversions \textcolor{red}{?} and more. The amplitude decay increases with the depth (z) as a function of \textbf{some tissue properties} ($a_0$) and \textbf{acoustic frequency)}(f) 


		\begin{equation}
		A(z) = A_0 e^{-a_0 f^{n}z}
		\end{equation}

		\begin{wbox}{The tradeoff}
			\textcolor{OliveGreen}{The higher the frequency, the better the imaging resolution } \\
			\textcolor{red}{But, the worse the penetration depth} 
		\end{wbox}
	
	\subsection*{Piezo-electric effect}
	The purpose of the ultrasound transducer is to convert between electrical signal and acoustic energy. It act as both speaker and a microphone. By first emitting a very short sound pulse and then listen to the returning echoes (for a longer time then the emitting time). It is only possible to to one at a time. 

	The piezo-electric effect is a property that certain solids can have (some crystals and ceramics). Expose the material for electricity and there will respond with mechanical stress (like a speaker). Swapping places and expose it for mechanical stress will produce an electrical current (like a microphone)

	\subsection*{Image production}
	Producing the image involves switching between Transmit (Tx) mode : active (speaker) and Receive (Rx) mode : passive (microphone). There are different modes for the imaging also:

		\begin{itemize}
			\item \textbf{A(mplitude)-mode:} one scan line
			\item \textbf{B(rightness)-mode:} 2D spatial images with multiple neighboring A-lines
			\item \textbf{M(otion)-mode:} sequences of A-modes in time, thus 2D (popular in cardiology) 
		\end{itemize}

	Beamforming is the technique used to form sensor arrays for directional transmission and reception. By adjusting a time delay between element it is possible to mimic the spatially focus of a curved transducer. 

	\textcolor{red}{insert US focus illustration here} 


	\subsection*{Image quality}
	Contrast is what assesses visibility, so by comparing the image appearance between two distinct regions we want to be able to distinguish between object and background. 


















	


