\documentclass[12pt,preprintnumbers,amsmath,amssymb,nofootinbib,superscriptaddress]{revtex4-1}
\usepackage[paperheight=12cm,top=0.5cm,bottom=0.5cm,left=1cm,right=1cm]{geometry}
\usepackage{xparse}
\NewDocumentCommand{\DIV}{om}{%
  \IfValueT{#1}{\setcounter{#2}{\numexpr#1-1\relax}}%
  \csname #2\endcsname
}
%\usepackage{pdftex}
\usepackage[latin1]{inputenc}
\usepackage{slashed}
\usepackage{amsmath}
\usepackage{textcomp}
\usepackage{amssymb}
\usepackage{amsfonts}
\usepackage{indentfirst}
\usepackage{color}
\usepackage[dvipsnames]{xcolor}
\usepackage{hyperref}
\usepackage{subcaption}
\usepackage{graphicx,graphics}
\graphicspath{{figures/}}
\usepackage[skins,theorems,many]{tcolorbox}
\tcbset{highlight math style={enhanced,
  colframe=red,colback=white,arc=0pt,boxrule=1pt}}
\usepackage[abs]{overpic}
\usepackage{xcolor,varwidth}
\usepackage[english]{babel}
\usepackage{blindtext,tikz}
\usetikzlibrary{calc}
\usepackage{fancyhdr}

\def\bibsection{\section*{}} %Removes black line above refrences

%--------------------------%
%----- COLOURED BOXES -----%
%--------------------------%

%Blue equation box
%\tcbhighmath[colframe=RoyalBlue!70!black,colback=RoyalBlue!25!white]{

%Orange equation box
%\tcbhighmath[colframe=BurntOrange!95!black,colback=BurntOrange!45!white]{

%Purple equation box
%\tcbhighmath[colframe=Purple!150!black,colback=Purple!30!white]{

%Green equation box
%\tcbhighmath[colframe=ForestGreen,colback=ForestGreen!25!white]{

%Grey equation box
%\tcbhighmath[colframe=Black,colback=Black!10!white]{

%Pink equation box
%\tcbhighmath[colframe=magenta,colback=magenta!20!white]{

%Blue and red equation box
%\tcbhighmath[frame style={left color=RoyalBlue!70!black,right color=Red!95!black},interior style={left color=RoyalBlue!35!white,right color=Red!50!white}]{

%Red and green equation box
%\tcbhighmath[frame style={left color=Red!95!black,right color=ForestGreen},interior style={left color=Red!50!white,right color=ForestGreen!25!white}]{

\begin{document}



\pagenumbering{gobble}


\vspace{10cm}

\title{Hyperspectral satellite imaging}


\author{Linus Falk}

\affiliation{Digital imaging systems - 1MD130}


\maketitle

\newpage

%-----EXAMPLE SLIDES-----%

\newpage
\DIV[1]{section}{Introduction}\label{Ueff}
\vspace{-0.2cm}\hrule

\vspace{0.5cm}

%What is it and what is it used for- images of system and example images the system creates

  \begin{itemize}
    \item Spectroscopy of reflected light from earth surface
    \begin{itemize}
      \item Passive technique
      \item Materials characterized by spectral signature (Absorption and reflection)
      \item Acquires images in many spectral bands so for each pixel a reflectance spectrum can be derived
      \item Important absorption features occur in the 400-2500 nm band (reflected solar radiation dominates natural EMS)
    \end{itemize}
  \end{itemize}

\vspace{1cm}


\newpage

\DIV[1]{section}{Introduction}\label{Ueff}
\vspace{-0.2cm}\hrule

\vspace{2cm}

%What is it and what is it used for- images of system and example images the system creates
  \begin{itemize}
    \item Hyperspectral imaging used in many fields 
      \begin{itemize}
        \item Ecosystem processes
        \item Surface mineralogy
        \item Water quality
        \item Soil type and erosion, 
        \item vegetation type and more...
      \end{itemize}
  \end{itemize}


\vspace{1cm}
\vspace{\fill}



\newpage
\DIV[1]{section}{Brief History}
\vspace{-0.2cm}
\hrule
\vspace{0.8cm}

\begin{minipage}{0.6\textwidth}

  \begin{itemize}
    \item +30 Years of hyperspectral satellite imaging
    \begin{itemize}
      \item Landsat-1 - \textbf{1972 NASA/JPL} (multispectral)
      \begin{itemize}
        \item Portable field reflectance spectrometers developed
      \end{itemize}
      \item For a long time was the spectral regions with atmospheric absorption seen as drawback
      \item Better algorithms and hardware made it possible to correct for this
      \item First commercial hyperspectral imaging systems for airborne use (DAIS 1989)
      \item Hyperion - \textbf{2000 NASA} %https://earthobservatory.nasa.gov/features/EO1/eo1_2.php
      \item EnMAP - \textbf{2022 DLR}
    \end{itemize}
  \end{itemize}

  %https://www.weather.gov/jetstream/absorb
  %https://www.l3harrisgeospatial.com/Support/Maintenance-Detail/ArtMID/13350/ArticleID/17333/Atmospheric-Windows-and-Optical-Sensors



\end{minipage}

\begin{minipage}{0.39\textwidth}



\end{minipage}
\vspace{\fill}



\newpage
\DIV[1]{section}{Use today \& limiting factors}
\vspace{-0.2cm}
\hrule
\vspace{0.8cm}

\begin{minipage}{0.6\textwidth}

\begin{itemize}
  \item Used in research
  \item Drawbacks 
  \begin{itemize}
    \item Not easy to deploy 
  \end{itemize}
\end{itemize}

\end{minipage}

\begin{minipage}{0.39\textwidth}



\end{minipage}

\vspace{\fill}
\centering


\newpage 
\DIV[2]{section}{References}
\vspace{-0.2cm}\hrule

\bibliography{Bibliography}

\vspace{\fill}
\centering





\end{document}