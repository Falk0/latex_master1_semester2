\documentclass[12pt,preprintnumbers,amsmath,amssymb,nofootinbib,superscriptaddress]{revtex4-1}
\usepackage[paperheight=12cm,top=0.5cm,bottom=0.5cm,left=1cm,right=1cm]{geometry}
\usepackage{xparse}
\NewDocumentCommand{\DIV}{om}{%
  \IfValueT{#1}{\setcounter{#2}{\numexpr#1-1\relax}}%
  \csname #2\endcsname
}
%\usepackage{pdftex}
\usepackage[latin1]{inputenc}
\usepackage{slashed}
\usepackage{amsmath}
\usepackage{textcomp}
\usepackage{amssymb}
\usepackage{amsfonts}
\usepackage{indentfirst}
\usepackage{color}
\usepackage[dvipsnames]{xcolor}
\usepackage{hyperref}
\usepackage{subcaption}
\usepackage{graphicx,graphics}
\graphicspath{{figures/}}
\usepackage[skins,theorems,many]{tcolorbox}
\tcbset{highlight math style={enhanced,
  colframe=red,colback=white,arc=0pt,boxrule=1pt}}
\usepackage[abs]{overpic}
\usepackage{xcolor,varwidth}
\usepackage[english]{babel}
\usepackage{blindtext,tikz}
\usetikzlibrary{calc}
\usepackage{fancyhdr}

\def\bibsection{\section*{}} %Removes black line above refrences


\begin{document}



\pagenumbering{gobble}


\vspace{10cm}

\title{Hyperspectral satellite imaging}


\author{Linus Falk}

\affiliation{Digital imaging systems - 1MD130}

\maketitle

\newpage

%-----EXAMPLE SLIDES-----%

\newpage
\DIV[1]{section}{Introduction}
\vspace{-0.2cm}
\hrule
\vspace{0.8cm}
\begin{minipage}{0.6\textwidth}

%What is it and what is it used for- images of system and example images the system creates

  \begin{itemize}
    \item Spectroscopy of reflected light from earth surface
    \begin{itemize}
      \item Passive technique
      \item Materials characterized by spectral signature (Absorption and reflection)
      \item Acquires images in many spectral bands so for each pixel a reflectance spectrum can be derived
      \item Important absorption features occur in the 400-2500 nm band (reflected solar radiation dominates natural EMS)
    \end{itemize}
  \end{itemize}

\end{minipage}
\vspace{\fill}

% ------------------------------------------------- %

\newpage
\DIV[1]{section}{Introduction}
\vspace{-0.2cm}
\hrule
\vspace{0.8cm}
\begin{minipage}{0.6\textwidth}

%What is it and what is it used for- images of system and example images the system creates
  \begin{itemize}
    \item Hyperspectral imaging used in many fields 
      \begin{itemize}
        \item Ecosystem processes
        \item Surface mineralogy
        \item Water quality
        \item Soil type and erosion, 
        \item vegetation type and more...
      \end{itemize}
  \end{itemize}

\end{minipage}
\vspace{\fill}

% ------------------------------------------------- %

\newpage
\DIV[1]{section}{Brief History}
\vspace{-0.2cm}
\hrule
\vspace{0.8cm}
\begin{minipage}{0.6\textwidth}
  \begin{itemize}
    \item +30 Years of hyperspectral satellite imaging
    \begin{itemize}
      \item Landsat-1 - \textbf{1972 NASA/JPL} (multispectral)
      \begin{itemize}
        \item Portable field reflectance spectrometers developed
      \end{itemize}
      \item For a long time was the spectral regions with atmospheric absorption seen as drawback
      \item Better algorithms and hardware made it possible to correct for this
      \item First commercial hyperspectral imaging systems for airborne use (DAIS \textbf{1989})
      \item EO1 - Hyperion sensor - \textbf{2000 NASA} %https://earthobservatory.nasa.gov/features/EO1/eo1_2.php
      \item EnMAP - \textbf{2022 DLR}
    \end{itemize}
  \end{itemize}

  %https://www.weather.gov/jetstream/absorb
  %https://www.l3harrisgeospatial.com/Support/Maintenance-Detail/ArtMID/13350/ArticleID/17333/Atmospheric-Windows-and-Optical-Sensors
\end{minipage}
\vspace{\fill}

% ------------------------------------------------- %

\newpage
\DIV[1]{section}{Use today \& limiting factors}
\vspace{-0.2cm}
\hrule
\vspace{0.8cm}

\begin{minipage}{0.6\textwidth}

\begin{itemize}
  \item Used in research
  \item Drawbacks 
  \begin{itemize}
    \item Not easy to deploy 
  \end{itemize}
\end{itemize}

\end{minipage}
\vspace{\fill}

% ------------------------------------------------- %

\newpage
\DIV[1]{section}{How is the image formed}
\vspace{-0.2cm}
\hrule
\vspace{0.8cm}

\begin{minipage}{0.6\textwidth}

\begin{itemize}
  \item Sun the source of energy 
  \item Sensor type:
  \begin{itemize}
    \item Diffraction Grating Spectrometers
    \item Prism Spectrometers
  \end{itemize}
  \item Imaging: 
  \begin{itemize}
    \item Push-broom
    \item Whiskbroom
  \end{itemize}
\end{itemize}

\end{minipage}
\vspace{\fill}

% ------------------------------------------------- %

\newpage
\DIV[1]{section}{What property of the sample is imaged?}
\vspace{-0.2cm}
\hrule
\vspace{0.8cm}
\begin{minipage}{0.6\textwidth}
\begin{itemize}
  \item Each material has a unique spectral characteristic
  \item Interaction radiation
  \begin{itemize}
    \item Absorbtion - material and wavelenght
    \item Reflection
    \item Transmission 
  \end{itemize}
\end{itemize}

\end{minipage}
\vspace{\fill}

% ------------------------------------------------- %

\newpage
\DIV[1]{section}{Resolution and sample size}
\vspace{-0.2cm}
\hrule
\vspace{0.8cm}
\begin{minipage}{0.6\textwidth}

\begin{itemize}
  \item Spatial resolution
  \begin{itemize}
    \item Field-of-view (\textbf{FOV})
    \item Instantaneous-field-of-view (\textbf{IFOV}) 
    \item Ground-projected instantaneous-field-of-view (\textbf{GIFOV})
    \begin{itemize}
      \item depends on the satellite elevation and varies with the viewing angle
    \end{itemize}
    \item Across-track (ACT) and along-track (ALT) resolution
    \begin{itemize}
      \item affected by integration time and smearing effects
    \end{itemize}
  \end{itemize}
\end{itemize}

\end{minipage}
\vspace{\fill}

% ------------------------------------------------- %

\newpage
\DIV[1]{section}{Resolution and sample size}
\vspace{-0.2cm}
\hrule
\vspace{0.8cm}
\begin{minipage}{0.6\textwidth}

\begin{itemize}
  \item Spectral resolution
  \begin{itemize}
    \item Portion of the \textbf{EMS} to which an instrument is sensitive
    \item Hyperspectral imaging - hundred of channels
  \end{itemize}
  \item Radiometric resolution
  \begin{itemize}
    \item Ability of the sensor to register differences in radiation
    \item Typically 8 and 12 bit, (EnMAP 14 bits)
  \end{itemize}
  \item Temporal resolution
  \begin{itemize}
    \item Time between two acquisitions
    \item Depends on satelite orbit 
    \item Vary greatly depending on cloud coverege
  \end{itemize}
\end{itemize}

\end{minipage}
\vspace{\fill}

% ------------------------------------------------- %

\newpage
\DIV[1]{section}{Resolution and sample size}
\vspace{-0.2cm}
\hrule
\vspace{0.8cm}
\begin{minipage}{0.6\textwidth}
\begin{itemize}
  \item Trade off
\end{itemize}

In the relationship of spatial and spectral
resolution, as one increases the other needs to decrease to gain the same amount
of photons on the detector. That means you either get an image with high spectral
but rather low spatial resolution (1), an image with high spatial but low spectral
resolution (2) or something in-between (both medium resolution). Sensor design tries
to avoid low SNR (3) and tries to achieve the best trade-off between spectral, spatial
resolution and high SNR for a certain application.

\end{minipage}
\vspace{\fill}

% ------------------------------------------------- %

\newpage
\DIV[1]{section}{Calibration and correction}
\vspace{-0.2cm}
\hrule
\vspace{0.8cm}
\begin{minipage}{0.6\textwidth}
% https://elib.dlr.de/124009/1/Delosreyes_L2A_DESIS_EnMAP_V02.pdf

\begin{itemize}
  \item 
  \item topographic 
  %https://www.enmap.org/data/doc/EN-PCV-TN-6007_Level_2A_Processor_Atmospheric_Correction_Land.pdf
  % https://www.enmap.org/data/doc/Science_Plan_EnMAP_2022_final.pdf
\end{itemize}
\end{minipage}
\vspace{\fill}

% ------------------------------------------------- %

\newpage
\DIV[1]{section}{Cost and limiting factors}
\vspace{-0.2cm}
\hrule
\vspace{0.8cm}
\begin{minipage}{0.6\textwidth}

\end{minipage}
\vspace{\fill}

% ------------------------------------------------- %

\newpage
\DIV[1]{section}{Variants and future use}
\vspace{-0.2cm}
\hrule
\vspace{0.8cm}
\begin{minipage}{0.6\textwidth}

\end{minipage}
\vspace{\fill}

% ------------------------------------------------- %

\newpage
\DIV[1]{section}{Title}
\vspace{-0.2cm}
\hrule
\vspace{0.8cm}
\begin{minipage}{0.6\textwidth}

\end{minipage}
\vspace{\fill}

% ------------------------------------------------- %
\newpage 
\DIV[2]{section}{References}
\vspace{-0.2cm}\hrule

\bibliography{Bibliography}

\vspace{\fill}
\centering





\end{document}


