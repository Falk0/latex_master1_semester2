\section{Chapter 1 - Graphics systems and models}
The field began +50 years ago with the display of a few lines on a CRT display and today we can genearte images that are indistinguiasable from photographs of real objects. The applications of computer graphics:

\begin{itemize}
	\item Display information
	\item Design
	\item Simulation and animation
	\item User interfaces
\end{itemize}

\subsection*{Display information}

Medical imaging poses an interesting and important data-analysis problem. CT MRI, ultrasound and PET generate three dimensional data that  must be subjected to algorithmic manipulation to provide any useful information for the user. 

\subsection*{Design}
Design problems are typical engineering problems were we start with a set of specifications and seek a cost-effective and estethic way do solve it. There are often not one solution. They are either overdetermined or underdetermined, there is no solution that satisfy all conditions or there are multiple ways to do it. The design process is therefor iterative and thanks to CAD programs its possible to generate imagaes through the process that are displayed with computer graphics. 

\subsection*{Simulations and animation}
Example of simulations are training pilots in simulators. VR has opened a new door for simulations and are also increasing.

\subsection*{User interfaces}
The interface with a PC is visual and become more and more advance and better looking, easier to use thanks to computer graphics.

\section{A graphic system}
The high level view of a graphics system:
\begin{itemize}
	\item Input device
	\item CPU
	\item GPU
	\item Memory
	\item Frame buffer
	\item Output devices
\end{itemize}


\subsection*{pixels and the framebuffer}
most modern graphics systems are raster based. The image we see on the output is an array of pixels. The pixels are stored in a part of the memory that is called the \textbf{frame buffer}. The resolution determines the detail that we can see. The \textbf{depth} determines how many colors that can be represented. 8 bit color depth give us 256 differente intensitys or colors. In a \textbf{full color system} there are 24 (or more) bits per pixel ($3 \times 8$) 

